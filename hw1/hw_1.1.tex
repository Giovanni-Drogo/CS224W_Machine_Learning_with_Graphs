\documentclass{article}
\usepackage{amsmath}
\usepackage{graphicx}
\usepackage{caption}
\usepackage{subcaption}

\begin{document}

\section*{Effect of Depth on Expressiveness}

For this sub-question, our update rule is:

\[
h_{v}^{k+1} = h_{v}^{k} + \sum_{i \in \mathcal{N}_{v}} h_{i}^{k}
\]

where \(\mathcal{N}_{v}\) is the neighbourhood of node \(v\), and \(k\) is the layer number. I will just use a recurrence instead of drawing graphs as I think that is much simpler. I will just use the number for the certain node. 

\subsection*{Layer \(k=0\):}
Does not work, because that means no propagation and these nodes have the same initial feature vector = \([1]\).

\subsection*{Layer \(k=1\):}
Also does not work, because these nodes have the same one-hop neighbourhood, which is the node \(5\).

\subsection*{Layer \(k=2\):}
Will also not work, because they have the same 2-hop neighbourhood (nodes \(5\), \(4\), and \(6\)), but I will show this case as an example. We need to compute \(h_T^{(2)}\) for both graphs, i.e.,

\[
h_T^{(2)} = h_T^{(1)} + \sum_{i \in N_T} h_i^{(1)}
\]

For the Left Graph:

\begin{align*}
h_T^{(2)} &= h_T^{(1)} + \sum_{i \in N_T} h_i^{(1)} \\
&= (h_T^{(0)} + h_5^{(0)}) + h_5^{(1)} \\
&= [2] + (h_5^{(0)} + h_4^{(0)} + h_T^{(0)} + h_6^{(0)}) \\
&= [6]
\end{align*}

This will be the same for the Right Graph.

\subsection*{Layer \(k=3\):}
Will work. We need to compute \(h_T^{(3)}\) for both graphs, i.e.,

\[
h_T^{(3)} = h_T^{(2)} + h_5^{(2)}
\]

For the Left Graph:

\begin{align*}
h_T^{(3)} &= h_T^{(2)} + h_5^{(2)} \\
&= [6] + (h_5^{(1)} + h_4^{(1)} + h_T^{(1)} + h_6^{(1)}) \\
&= [6] + (h_5^{(0)} + h_4^{(0)} + h_T^{(0)} + h_6^{(0)}) \\
&\quad + (h_4^{(0)} + h_5^{(0)} + h_3^{(0)}) + (h_T^{(0)} + h_5^{(0)}) \\
&\quad + (h_5^{(0)} + h_6^{(0)} + h_7^{(0)}) \\
&= [18]
\end{align*}

For the Right Graph:

\begin{align*}
h_T^{(3)} &= h_T^{(2)} + h_5^{(2)} \\
&= [6] + (h_5^{(1)} + h_4^{(1)} + h_T^{(1)} + h_6^{(1)}) \\
&= [6] + (h_5^{(0)} + h_4^{(0)} + h_T^{(0)} + h_6^{(0)}) \\
&\quad + (h_4^{(0)} + h_5^{(0)} + h_3^{(0)}) + (h_T^{(0)} + h_5^{(0)}) \\
&\quad + (h_A^{(0)} + h_6^{(0)} + h_7^{0} + h_8^{0}) \\
&= [19]
\end{align*}

The only difference was Node \(6\) having vs. not having a connection to Node \(8\).

\end{document}